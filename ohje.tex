\documentclass[12pt,a4paper]{scrartcl}
\usepackage[finnish]{babel}
\usepackage[T1]{fontenc}
\usepackage[utf8]{inputenc}
\usepackage{zref-lastpage,zref-user}
\usepackage{fancyhdr}
\frenchspacing
\lhead{Sukijan asennus- ja käyttöohje (Copyright © 2011--2015 Hannu Väisänen)}
\rhead{\thepage(\zpageref{LastPage})}
\cfoot{}
\begin{document}
\pagestyle{fancy}
\setlength{\parindent}{0pt}
\setlength{\parskip}{1ex plus 0.5ex minus 0.2ex}
\section*{Sukijan asennus- ja käyttöohje}

Sukija on Javalla kirjoitettu ohjelma suomenkielisten tekstien
indeksointiin.

Sukija analysoi sanat morfologisesti, muuttaa sanat perusmuotoon (joka
on sanakirjoissa) ja indeksoi perusmuodot, jotta sanan kaikki
taivutusmuodot löytyvät vain perusmuotoa etsimällä.

Sukija tallentaa perusmuodot Solr:n tietokantaan, josta niitä voi
etsiä Solr:n käyttöliittymän kautta.

Sukija osaa indeksoida kaikkia niitä tiedostomuotoja, joita Apache
Tika \\(http://tika.apache.org/) osaa lukea.

\subsection*{Mitä tarvitaan ja mistä ne saa?}

\begin{itemize}

\item Sukija:
      \verb=https://github.com/ahomansikka/sukija= \\
      Koska luet tätä tekstiä, olet jo imuroinut tämän. (-:

\item Suomi-Malaga: Se on corevoikossa
      (\verb=https://github.com/voikko/corevoikko=) \\
      hakemistossa suomimalaga.

\item Apache Solr 5.0.0:
      \verb=http://lucene.apache.org/solr/= \\
      Tässä dokumentissa Solr:sta käytetään nimeä solr-x.y.z,
      missä x.y.z on version numero, esimerkiksi 5.0.0.

\item Ubuntun paketit \verb=libmalaga7= ja \verb=maven=.
\end{itemize}

Lisäksi Sukija tarvitsee erinäisiä jar-tiedostoja, mutta Maven imuroi
ne verkosta automaagisesti.

Sukijaa voi käyttää myös Voikon Java-version kanssa. Tällöin tarvitaan
Ubuntun paketti \verb=libvoikko1=.

Tämä asennusohje olettaaa, että corevoikko ja apache-solr ovat hakemistoissa \\
\verb=$HOME/Lataukset/corevoikko= ja
\verb=$HOME/Lataukset/solr/solr-x.y.z=

Jos ne ovat jossain muualla, tiedoston \verb=Makefile= alussa olevia
muuttujia \verb=SOLR_HOME= ja \verb=JETTY_CONTEXTS_DIR= pitää muuttaa
vastaavasti.

\newpage
\subsection*{Ohjelman rakenne}

Sukijassa on neljä osaa:

\begin{itemize}
\item sukija-core     Java-luokkia, joita muut ohjelman osat tarvitsevat.
\item sukija-malaga   Solr:n liitännäinen, joka käyttää Suomi-Malagan
                      Sukija-versiota \\muuttamaan sanat perusmuotoon.
\item sukija-voikko   Solr:n liitännäinen, joka käyttää Voikkoa
                      (Malaga- tai Vfst"-morfologiaa) muuttamaan sanat perusmuotoon.
\item sukija-ui       Javalla tehty käyttöliittymä (keskeneräinen).
\end{itemize}


\subsection*{Suomi-Malagan asentaminen}

Suomi-Malagasta on kaksi versiota, Voikko-versio on tarkoitettu
oikolukuun ja Sukija tiedostojen indeksointiin. Sukija-versio
käännetään komennolla

\begin{verbatim}
cd $HOME/Lataukset/corevoikko/suomimalaga
make sukija
\end{verbatim}

Tee alihakemisto \verb=$HOME/.sukija= ja kopioi sinne tiedostot \\
\verb=suomimalaga/sukija/{suomi.*_l,suomi.pro}=

Myös Voikko-versiota voi käyttää indeksointiin, kun sen kääntää ja
asentaa komennoilla

\begin{verbatim}
cd $HOME/Lataukset/corevoikko/suomimalaga
make voikko-sukija
make voikko-install DESTDIR=~/.voikko
\end{verbatim}

\verb|DESTDIR| voi olla myös joitan muuta kuin \verb|~/.voikko|.

Versioiden erot ovat siinä, että Sukija"-versio tunnistaa myös vanhoja
taivutusmuotoja ja sanoja sekä yleisiä kirjoitusvirheitä.


\subsection*{Sukijan kääntäminen ja asentaminen, Solr:n konfigurointi}

Ensin käännetään Sukija komennolla

\verb=mvn package=

Komento imuroi netistä tarvitsemansa Javan jar-paketit eli ensimmäinen
kääntäminen saattaa kestää kauan. Erityisen kauan se kestää, jos et
ole aiemmin käyttänyt mavenia.

Testien aikana tulee virheilmoitus

SLF4J: Class path contains multiple SLF4J bindings.

Siitä ei tarvitse välittää.

\bigskip Toisessa vaiheessa asetetaan Solr:n konfigurointitiedostoon
\verb=schema.xml= saneistajaluokka (ulkomaankielellä tokenizer), joka
lukee sanat tiedostoista, ja morfologialuokka, joka muuttaa sanat
perusmuotoon. Komennolla \verb|make ____-schema| on viisi eri
vaihtoehtoa:

\begin{tabular}{@{}ll}
Komento                              & Morfologialuokka \\
\verb=make malaga-schema=            & MalagaMorphologyFilterFactory \\
\verb=make malaga-suggestion-schema= & MalagaMorphologySuggestionFilterFactory \\
\verb=make voikko-schema=            & VoikkoMorphologyFilterFactory \\
\verb=make voikko-suggestion-schema= & VoikkoMorphologySuggestionFilterFactory \\
\verb=make vfst-schema=              & VoikkoMorphologyFilterFactory \\
\verb=make vfst-suggestion-schema=   & VoikkoMorphologySuggestionFilterFactory \\
\verb=make debug-schema=             & \\
\verb=make debug-vfst-schema=
\end{tabular}

Vfst on Voikko, joka käyttää uutta vfst-morfologiaa.

Komentoja \verb=make debug-schema= ja \verb=make debug-vfst-schema=
käytetään vain Sukijan kehittämiseen.

Saneistajan oletusarvo on FinnishTokenizerFactory, joka tulee Sukijan
mukana, mutta sen voi vaihtaa muuttujalla TOKENIZER\_FACTORY
esimerkiksi näin:

\verb|make voikko-schema TOKENIZER_FACTORY=solr.StandardTokenizerFactory|

\verb=____FilterFactory= ja \verb=____SuggestionFilterFactory= eroavat
toisistaan siten, että jos morfologialuokka ei tunnista sanaa,
\verb=Suggestion="-luokissa sanaan tehdään muutoksia (esimerkiksi
muutetaan w v:ksi) ja tunnistusta yritetään uudestaan. Tämä ei ole
sama asia kuin Voikon oikeinkirjoituksen korjausehdotukset!

\verb=Suggestion="-luokat konfiguroidaan tiedostossa
\verb|finnish-suggestion.xml|. Katso sivu \zpageref{suggestionConfiguration}.

\bigskip
Indeksoitavat tiedostot asetetaan tiedostossa \verb|data-config.xml|.

Katso \\
\verb|http://wiki.apache.org/solr/DataImportHandler#FileListEntityProcessor| \\
ja
\verb|http://wiki.apache.org/solr/TikaEntityProcessor|.

Tärkeimmät konfiguroitavat parametrit ovat:

\begin{itemize}
\item baseDir
Hakemisto, jossa ja jonka alihakemistoissa tiedostot ovat.

\item fileName
Säännöllinen lauseke, joka valitsee indeksoitavat tiedostot.

\item excludes
Säännöllinen lauseke, joka valitsee tiedostot, joita ei indeksoida.
\end{itemize}

Komento \verb|make install| asettaa näiden oletusarvoiksi

baseDir \verb|$HOME/Asiakirjat|

fileName \verb|.*| eli kaikki tiedostot indeksoidaan.

excludes \\
{\footnotesize \verb+(?u)(?i).*[.](au|bmp|bz2|class|gif|gpg|gz|jar|jpg|jpeg|m|o|png|tif|tiff|wav|zip)$+}


Näitä voidaan muuttaa parametreilla \verb|BASE_DIR|,
\verb|FILE_NAME| ja
\verb|EXCLUDES|. Esimerkiksi

\verb|make install BASE_DIR=/usr/local/data|

Säännöllisten lausekkeiden syntaksi on sama kuin Javan luokassa \\
\verb=java.util.regex.Pattern=.

Indeksoitavien tiedostojen asettamisen lisäksi komento
\verb=make install= kopioi hakemistossa \verb=conf= olevat Solr:n ja
Sukijan tarvitsemat tiedostot oikeisiin paikkoihin.

Tiedostot
\verb=data-config.xml=,
\verb=suggestion.txt= ja
\verb=synonyms.txt=
kopioidaan hakemistoon \verb=$HOME/.sukija=
ja tiedostot
\verb=schema.xml=,
\verb=solrconfig.xml=, \\
\verb=sukija-context.xml=,
\verb=sukija.xsl= ja
alihakemisto \verb=velocity=
niihin hakemistoihin, joista Solr löytää ne.


\subsection*{Solr:n käynnistäminen}

\begin{verbatim}
cd $HOME/Lataukset/solr/solr-x.y.z/example
../bin/solr start
../bin/solr create -c sukija
\end{verbatim}


Solr:n käynnistymisen voi varmistaa selaimessa menemällä verkko-osoitteeseen

\verb|http://localhost:8983/solr/|


\subsection*{Solr:n loki}

Solr:n lokitulostus (\verb=http://wiki.apache.org/solr/SolrLogging=)
konfiguroidaan tiedostossa
\verb=solr-x.y.z/example/resources/log4j.properties=
Mahdollisimman suuren lokitulostuksen saa lisäämällä tämän tiedoston
loppuun rivit

\begin{verbatim}
log4j.logger.peltomaa.sukija.finnish.HVTokenizer = ALL
log4j.logger.peltomaa.sukija.morphology.MorphologyFilter = ALL
log4j.logger.peltomaa.sukija.suggestion.Suggestion = ALL
log4j.logger.peltomaa.sukija.suggestion.SuccessFilter = ALL
log4j.logger.peltomaa.sukija.suggestion.SuggestionFilter = ALL
log4j.logger.peltomaa.sukija.malaga.MalagaMorphology = ALL
log4j.logger.peltomaa.sukija.voikko.VoikkoMorphology = ALL
log4j.logger.peltomaa.sukija.voikko.VoikkoMorphologySuggestionFilterFactory = ALL
log4j.logger.peltomaa.sukija.hyphen.HyphenFilter = ALL
\end{verbatim}

Tuossa ovat kaikki Sukijan luokat, joissa on lokitulostus.

Tällöin tulostus on paljon suurempi kuin indeksoitavat tiedostot (-:,
mutta kaikkia ei tietenkään tarvitse lisätä, riittää kun laittaa
luokat \verb=MorphologyFilter= sekä \\
\verb=SuggestionFilter= tai \verb=SuccessFilter=.

Luokkia \verb=SuggestionFilter= ja \verb=SuccessFilter= ei pidä
käyttää yhtaikaa.

Luokkaa \verb=HVTokenizer= ei tarvitse laittaa, jos ei käytä tätä
saneistajaa, ja luokkia \\
\verb=MalagaMorphology= ja \verb=VoikkoMorphology= Sukija ei käytä
yhtaikaa. \verb=SuggestionFilter= tuottaa lokitulostuksen kaikista
yrityksistä muuttaa sana perusmuotoon, \verb=Suggestion= kertoo
erikseen, mikä muunnos sai aikaan sanan tunnistamisen (katso sivu
\zpageref{suggestionConfiguration}).

Lokitulostuksen eri tasot (\verb=ALL=, jne) voi katsua luokan
\verb=org.apache.log4j.Level= dokumentoinnista.

Lokitulostus menee tiedostoon \verb=solr-x.y.z/example/logs/solr.log=


\subsection*{Indeksointi}

Tiedostot indeksoidaan menemällä osoitteeseen

\verb|http://localhost:8983/solr/#/sukija/dataimport?command=full-import|

Enemmän tai vähemmän pitkän ajan kuluttua indeksoinnin lopputuloksen
voi katsoa osoitteesta

\verb|http://localhost:8983/solr/dataimport|


\subsection*{Tietojen etsiminen}

Sanoja etsitään menemällä osoitteeseen
\verb=http://localhost:8983/solr/browse=

Etsittävien sanojen tulee olla perusmuodossa. Etsittäessä sanoja ei
muuteta perusmuotoon siksi, että yhden sanan perusmuoto voi olla
toisen sanan taivutusmuoto. Paras esimerkki tästä on ''alusta'', joka
on sanojen ''alusta'', ''alustaa'', ''alku'', ''alunen'' ja ''alus''
taivutusmuoto. Tällöin herää kysymys, mitä sanaa pitää etsiä, vai
etsitäänkö kaikkia?


Eri tavalla muotoillun tulostuksen saa osoitteesta \\
\verb=http://localhost:8983/solr/select=

Esimerkiksi sanaa \verb=sana= etsitään näin:

\verb|http://localhost:8983/solr/select?q=sana|

Tämän tulostuksen ulkonäköä voi muuttaa muuttamalla Sukijan mukana
tulevaa tiedostoa \verb=conf/sukija.xsl=.


\subsection*{Tiedoston finnish-suggestion.xml konfigurointi}
\zlabel{suggestionConfiguration}

Tiedosto \verb|finnish-suggestion.xml| pitää konfiguroida erikseen
Sukijalle ja Voikolle. Nykyinen konfiguraatio on tehty Voikolle ja sen
vfst"-morfologialle.

Tässä vaiheessa kaikki indeksoitavista tiedostoista luetut sanat on
muutettu pieniksi kirjaimiksi eli tiedostossa
\verb|finnish-suggestion.xml| olevat erisnimetkin pitää kirjoittaa
pienellä alkukirjaimella.

Konfiguroititiedoston formaatti on määritelty tiedostossa
\verb=sukija-core/src/main/xsd/SuggestionInput.xsd=.


%%%%%%%%%%%%%%%%%%%%%%%%%%%%%%%%%%%%%%%%%%%%%%%%%%%


\verb=compoundWordEnd= tunnistaa yhdyssanan, jos se loppuu tiettyyn sanaan. Esimeriksi

\begin{verbatim}
  <compoundWordEnd>
    <input>jo(k[ie]|e) joki</input>
  </compoundWordEnd>
\end{verbatim}

tunnistaa esimerkiksi merkkijonon ''aatsajoelle''. Tällä tavalla
voidaan tunnistaa paikannimiä, jotka eivät ole sanastossa.

Jokaisessa argumentissa on kaksi osaa. Ensimmäinen on säännöllinen
lauseke ja toinen jonkin sanan perusmuoto. Tunnistettaessa merkkijono
katkaistaan siitä kohdasta, josta säännöllinen lauseke alkaa, ja jos
merkkijonon loppuosan perusmuoto on argumentin toinen osa,
perusmuotona palauteaan merkkijonon alkuosa + argumentin toinen osa.

Esimeriksi ''aatsajoelle'' jaetaan kahtia osiin ''aatsa'' ja
''joelle'', ja koska merkkijonon ''joelle'' perusmuoto on ''joki'',
merkkijonon ''aatsajoelle'' perusmuodoksi tulee ''aatsajoki''.

%%%%%%%%%%%%%%%%%%%%%%%%%%%%%%%%%%%%%%%%%%%%%%%%%%%


Konfiguroinnissa on komennon nimi, jota voi seurata argumentteja,
esimerkiksi

\begin{verbatim}
  <suggestion name = "CompoundWordEnd">
    <argument>aukio aukio</argument>
  </suggestion>
\end{verbatim}


Komentojen nimet tulevat siitä,
että ne on toteutettu samannimisinä Java-luokkina tai ainakin melkein:
\verb|Apostrophe| on toteutettu luokassa \verb|ApostropheSuggestion|
ja niin edelleen.


\bigskip
\verb|<suggestion name = "Apostrophe"/>|

Poistaa sanasta heittomerkin ja yrittää tunnistaa sanan sen jälkeen.
Jos tunnistaminen ei onnistu, poistaa sanasta heittomerkin ja kaikki
sen jälkeiset merkit ja palauttaa jäljelle jääneet merkit sanan
perusmuotona. Esimerkiksi yrittää tunnistaa merkkijonon
\verb|centime'in| muodossa \verb|centimein|. Jos tunnistaminen ei
onnistu, palauttaa merkkijonon \verb|centime|.



\bigskip
\verb|Char| Muuttaa sanassa olevat merkit toiseksi. Tämä vastaa Unixin komentoa \verb|tr|.

Esimerkiksi

\begin{verbatim}
  <suggestion name = "Char">
    <argument>pk</argument>
    <argument>bg</argument>
  </suggestion>
\end{verbatim}

(1) muuttaa p:t b:iksi, jättää k:t ennalleen,
(2) muuttaa k:t g:iksi, jättää p:t ennalleen, sekä
(3) muuttaa p:t b:iksi ja k:t g:iksi.

Siis jos tiedostosta luettu sana on ''piolokia'', komento yrittää
tunnistaa sanat ''piolokia'', ''biolokia'', ''piologia'' ja ''biologia''.


\bigskip
\verb|<suggestion name = "Length3"/>|

poistaa kolmesta peräkkäisestä samasta kirjaimesta
yhden. \\ Esimerkiksi ''kauttta'' => ''kautta''.


\bigskip

\verb|Regex| muuttaa säännöllisen lausekkeen. Esimerkiksi


\begin{verbatim}
  <suggestion name = "Regex">
    <argument>ai(j)[eou]  </argument>
    <argument>C[ae](hi)C  i</argument> 
    <argument>true</argument>
  </suggestion>
\end{verbatim}

Ensimmäinen argumentti \verb|Regex "ai(j)[eou]" ""| poistaa
j"-kirjaimen muun muassa sanoista ''aijemmin'', ''aijomme'' ja
''kaijutin''.

Toinen argumentti \verb|C[ae](hi)C i|
poistaa h"-kirjaimet muun muassa sanoista ''ainahinen'' ja
''etehinen''

Viimeinen argumentti on joko \verb|true|, jolloin kokeillaan kaikkia
argumentteina olevia säännöllisiä lausekkeita ja palautetaan kaikki
tunnistetut sanat, tai \verb|false|, jolloin lopetetaan, kun
ensimmäinen säännöllinen lauseke on tunnistettu.

Säännöllisessä lausekkeessa voi käyttää kirjainta \verb|C|
tarkoittamaan konsonantteja ja kirjainta \verb|V| tarkoittamaan
vokaaleja.

Kirjaimia \verb|C| ja \verb|V| lukuun ottamatta säännöllisten
lausekkeiden syntaksi on sama kuin Javan luokassa
\verb=java.util.regex.Pattern=. Muita isoja kirjaimia ei tule
käyttää säännöllisissä lausekkeissa, koska tässä vaiheessa kaikki
tiedostoista luetut sanat on muutettu pieniksi kirjaimiksi, siis myös
isokirjaimiset lyhenteet ja erisnimien alkukirjaimet.


\bigskip
\verb|String| muuttaa merkkijonon toiseksi. Esimerkiksi

\begin{verbatim}
  <suggestion name = "String">
    <argument>tsh ts</argument>
    <argument>sydämm sydäm</argument>
  </suggestion>
\end{verbatim}

muuttaa merkkijonon ''mantshuria'' merkkijonoksi ''mantsuria'' ja
merkkijonon ''sydämmellisesti'' merkkijonoksi ''sydämellisesti''.

\bigskip
Näitä komentoja voi antaa mielivaltaisen paljon missä tahansa
järjestyksessä, ja \\ \verb|____SuggestionFilterFactory| palauttaa
perusmuotona ensimmäisen tunnistamansa muodon. Jos mitään ehdotusta ei
tunnisteta, \verb|____SuggestionFilterFactory| palauttaa alkuperäisen
merkkijonon.


\subsection*{Javalla kirjoitettu käyttöliittymä}

Kun tiedostot on indeksoitu, niitä voidaan tutkia myös komennolla

\verb|java -jar sukija-ui/target/sukija-ui-1.0.jar|

\end{document}
