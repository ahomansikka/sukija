\documentclass[12pt,a4paper]{scrartcl}
\usepackage[finnish]{babel}
\usepackage[T1]{fontenc}
\usepackage[utf8]{inputenc}
\usepackage{zref-lastpage,zref-user}
\usepackage{fancyhdr}
\usepackage{a4wide}
\frenchspacing
\lhead{Sukijan asennus- ja käyttöohje (Copyright © 2011--2015 Hannu Väisänen)}
\rhead{\thepage(\zpageref{LastPage})}
\cfoot{}
\begin{document}
\pagestyle{fancy}
\setlength{\parindent}{0pt}
\setlength{\parskip}{1ex plus 0.5ex minus 0.2ex}
\section*{Sukijan asennus- ja käyttöohje}

Sukija on Javalla kirjoitettu ohjelma suomenkielisten tekstien
indeksointiin.

Sukija analysoi sanat morfologisesti, muuttaa sanat perusmuotoon (joka
on sanakirjoissa) ja indeksoi perusmuodot, jotta sanan kaikki
taivutusmuodot löytyvät vain perusmuotoa etsimällä.

Sukija tallentaa perusmuodot Solr:n tietokantaan, josta niitä voi
etsiä Solr:n käyttöliittymän kautta.

Sukija osaa indeksoida kaikkia niitä tiedostomuotoja, joita Apache
Tika \\(\verb=http://tika.apache.org/=) osaa lukea.

\subsection*{Mitä tarvitaan ja mistä ne saa?}

\begin{itemize}

\item Sukija:
      \verb=https://github.com/ahomansikka/sukija= \\
      Koska luet tätä tekstiä, olet jo imuroinut tämän. (-:

\item Suomi-Malaga: Se on corevoikossa
      (\verb=https://github.com/voikko/corevoikko=) \\
      hakemistossa suomimalaga.

\item Apache Solr 5.0.0:
      \verb=http://lucene.apache.org/solr/= \\
      Tässä dokumentissa Solr:sta käytetään nimeä solr-x.y.z,
      missä x.y.z on version numero, esimerkiksi 5.0.0.

\item Ubuntun paketit \verb=libmalaga7= ja \verb=maven=.
\end{itemize}

Lisäksi Sukija tarvitsee erinäisiä jar-tiedostoja, mutta Maven imuroi
ne verkosta automaagisesti.

Sukijaa voi käyttää myös Voikon Java-version kanssa. Tällöin tarvitaan
Ubuntun paketti \verb=libvoikko1=.

Tämä asennusohje olettaaa, että corevoikko ja apache-solr ovat hakemistoissa \\
\verb=$HOME/Lataukset/corevoikko= ja
\verb=$HOME/Lataukset/solr/solr-x.y.z=

Jos ne ovat jossain muualla, tiedoston \verb=Makefile= alussa olevaa
muuttujaa \verb=SOLR= ja tiedoston \verb=asenna.sh= alussa olevia
muuttujia pitää muuttaa vastaavasti.


\newpage
\subsection*{Ohjelman rakenne}

Sukijassa on neljä osaa:

\begin{itemize}
\item sukija-core     Java-luokkia, joita muut ohjelman osat tarvitsevat.
\item sukija-malaga   Solr:n liitännäinen, joka käyttää Suomi-Malagan
                      Sukija-versiota \\muuttamaan sanat perusmuotoon.
\item sukija-voikko   Solr:n liitännäinen, joka käyttää Voikkoa
                      (Malaga- tai Vfst"-morfologiaa) muuttamaan sanat perusmuotoon.
\item sukija-ui       Javalla tehty käyttöliittymä (keskeneräinen).
\end{itemize}


\subsection*{Suomi-Malagan asentaminen}

Suomi-Malagasta on kaksi versiota, Voikko-versio on tarkoitettu
oikolukuun ja Sukija tiedostojen indeksointiin. Sukija-versio
käännetään komennolla

\begin{verbatim}
cd $HOME/Lataukset/corevoikko/suomimalaga
make sukija
\end{verbatim}

Tee alihakemisto \verb=$HOME/.sukija= ja kopioi sinne tiedostot \\
\verb=suomimalaga/sukija/{suomi.*_l,suomi.pro}=

Myös Voikko-versiota voi käyttää indeksointiin, kun sen kääntää ja
asentaa komennoilla

\begin{verbatim}
cd $HOME/Lataukset/corevoikko/suomimalaga
make voikko-sukija
make voikko-install DESTDIR=~/.voikko
\end{verbatim}

\verb|DESTDIR| voi olla myös joitan muuta kuin \verb|~/.voikko|.

Versioiden erot ovat siinä, että Sukija"-versio tunnistaa myös vanhoja
taivutusmuotoja ja sanoja sekä yleisiä kirjoitusvirheitä.



\subsection*{Sukijan kääntäminen ja asentaminen}

Ensin käännetään ja asennetaan Sukija komennolla

\verb=mvn install=

Komento imuroi netistä tarvitsemansa Javan jar-paketit eli ensimmäinen
kääntäminen saattaa kestää kauan. Erityisen kauan se kestää, jos et
ole aiemmin käyttänyt mavenia.

Sukijan jar-tiedostot asennetaan maven-hakemistoon \verb|${HOME}/.m2|

Testien aikana tulee virheilmoitus

SLF4J: Class path contains multiple SLF4J bindings.

Siitä ei tarvitse välittää.


\subsection*{Solr:n konfigurointi (1)}

Solr:n parametrit asetetaan tiedostossa \verb|sukija.properties|,
jossa on myös tarvittavat ohjeet.

Sen jälkeen kofiguroidaan Solr komennolla \verb|make asenna|

Jos asentamisen jäkeen muuttaa tiedostoa
\verb|sukija.properties| tai \\
\verb|conf2/sukija-context.xml|
muutokset voi viedä Solr:iin komennolla \verb|make päivitä|

\verb=make asenna= tekee saman kuin Solr:n komennot \\
\verb=bin/solr start; bin/solr create -c sukija -d conf=, mutta lisäksi se
kopioi Solr:n tarvitsemat jar"-tiedostot hakemistoon \\
\verb=$HOME/Lataukset/solr/solr-x.y.z/server/solr/sukija/lib=


\subsection*{Solr:n käynnistäminen}

\verb|make asenna| käynnistää Solr:n.

Myöhemmin sen voi käynnistää komennolla \\
\verb|~/Lataukset/solr/solr-x.y.z/bin/solr start|


Solr:n käynnistymisen voi varmistaa selaimessa menemällä
verkko"-osoitteeseen \\
\verb|http://localhost:8983/solr/|


\subsection*{Solr:n konfigurointi (2)}

Tämä voidaan tehdä vasta sen jälkeen kun on tehty Solr:n konfigurointi (1).

Jos olet asentanut Solr:n skriptillä \verb=install_solr_service.sh=
(katso Apache Solr Reference Guide kohta
Service Installation Script),
Solr konfiguroidaan Sukijaa varten näin.

Skripti asentaa Solr:n hakemistoihin \verb=/opt/solr-5.0.0= ja
\verb=/var/solr=. Niiden omistajaksi tehdään käyttäjä \verb=solr=,
mutta jos käytät lippua \verb=-u=, voi asettaa hakemistojen
omistajaksi itsesi:

\verb=sudo bash ./install_solr_service.sh solr-5.0.0.tgz -u username=

missä username on oma käyttäjätunnuksesi. Nämä asennusohjeet
olettavat, että olet tehnyt näin, että olet asentanut Solr:n
oletushakemistoihin ja että olet Sukijan päähakemistossa eli
hakemistossa, jossa on tiedosto ohje.tex.

Asennus konfiguroidaan komennolla \verb=make service=
joka yksinkertaisesti kopioi hakemiston
\verb=${HOME}/Lataukset/solr/solr-5.0.0/server/solr/sukija=
hakemistoon \\
\verb=/var/solr/data= ja tiedoston
\verb=conf2/sukija-context.xml= hakemistoon \\
\verb=/opt/solr/server/contexts=.


Jos asentamisen jäkeen muuttaa tiedostoa
\verb|sukija.properties| tai \\
\verb|conf2/sukija-context.xml|
muutokset viedään Solr:iin komennolla \\
\verb=make service-update=

Seuraavaksi mennään osoitteeseen \verb=http://localhost:8983/solr/=
Vasemmalla paneelin alalaidassa lukee 

\begin{verbatim}
No cores available
Go and create one
\end{verbatim}

Mennään sinne ja kirjotetaan kohtaan \verb=name= ja \verb=instanceDir=
''sukija'' (mutta ilman lainausmerkkejä) ja napsautetaan kohtaa
\verb=Add Core=.

Nyt vasempaan paneeliin alalaitaan pitäisi tulla mahdollisuus valita
indeksi (Solr käyttää siitä nimeä \verb=core=) sukija. Sen jälkeen
voimme ruveta indeksoimaan. Katso sivu \zpageref{indexing}.


\subsection*{Solr:n loki}

Solr:n lokitulostus (\verb=http://wiki.apache.org/solr/SolrLogging=)
konfiguroidaan tiedostossa
\verb=solr-x.y.z/server/resources/log4j.properties= tai \\
\verb=/var/solr/log4j.properties=

Mahdollisimman suuren lokitulostuksen saa lisäämällä tämän tiedoston
loppuun rivit

%log4j.logger.peltomaa.sukija.finnish.FinnishTokenizer = ALL

{\footnotesize
\begin{verbatim}
log4j.logger.peltomaa.sukija.finnish.HVTokenizer = ALL
log4j.logger.peltomaa.sukija.hyphen.HyphenFilter = ALL
log4j.logger.peltomaa.sukija.malaga.MalagaMorphology = ALL
log4j.logger.peltomaa.sukija.morphology.MorphologyFilter = ALL
log4j.logger.peltomaa.sukija.suggestion.Suggestion = ALL
log4j.logger.peltomaa.sukija.suggestion.SuggestionFilter = ALL
log4j.logger.peltomaa.sukija.suggestion.SuggestionParser = ALL
log4j.logger.peltomaa.sukija.voikko.VoikkoMorphologyFilterFactory = ALL
log4j.logger.peltomaa.sukija.voikko.VoikkoMorphology = ALL
log4j.logger.peltomaa.sukija.voikko.VoikkoMorphologySuggestionFilterFactory = ALL
\end{verbatim}
}

Tuossa ovat kaikki Sukijan luokat, joissa on lokitulostus.

Tällöin tulostus on paljon suurempi kuin indeksoitavat tiedostot (-:,
mutta kaikkia ei tietenkään tarvitse lisätä, riittää kun laittaa
luokat \verb=MorphologyFilter= sekä \\
\verb=SuggestionFilter=.

Luokkaa \verb=HVTokenizer= ei tarvitse laittaa, jos ei käytä tätä
saneistajaa, ja luokkia \\
\verb=MalagaMorphology= ja \verb=VoikkoMorphology= Sukija ei käytä
yhtaikaa. \verb=SuggestionFilter= tuottaa lokitulostuksen kaikista
yrityksistä muuttaa sana perusmuotoon, \verb=Suggestion= kertoo
erikseen, mikä muunnos sai aikaan sanan tunnistamisen (katso sivu
\zpageref{suggestionConfiguration}).

Lokitulostuksen eri tasot (\verb=ALL=, jne) voi katsua luokan
\verb=org.apache.log4j.Level= dokumentoinnista.

Lokitulostus menee tiedostoon \\
\verb=$HOME/Lataukset/solr/solr-x.y.z/server/logs/solr.log= tai \\
\verb=/var/solr/logs/solr.log=


\subsection*{Indeksointi}
\zlabel{indexing}

Tiedostot indeksoidaan menemällä osoitteeseen \\
\verb|http://localhost:8983/solr/sukija/dataimport?command=full-import|

Enemmän tai vähemmän pitkän ajan kuluttua indeksoinnin lopputuloksen
voi katsoa osoitteesta
\verb|http://localhost:8983/solr/sukija/dataimport|


\subsection*{Tietojen etsiminen}

Sanoja etsitään menemällä osoitteeseen \\
\verb=http://localhost:8983/solr/sukija/browse=

Etsittävien sanojen tulee olla perusmuodossa. Etsittäessä sanoja ei
muuteta perusmuotoon siksi, että yhden sanan perusmuoto voi olla
toisen sanan taivutusmuoto. Paras esimerkki tästä on ''alusta'', joka
on sanojen ''alusta'', ''alustaa'', ''alku'', ''alunen'' ja ''alus''
taivutusmuoto. Tällöin herää kysymys, mitä sanaa pitää etsiä, vai
etsitäänkö kaikkia?


Eri tavalla muotoillun tulostuksen saa osoitteesta \\
\verb=http://localhost:8983/solr/sukija/select=

Esimerkiksi sanaa \verb=sana= etsitään näin: \\
\verb|http://localhost:8983/solr/sukija/select?q=sana|

Tämän tulostuksen ulkonäköä voi muuttaa muuttamalla Sukijan mukana
tulevaa tiedostoa \verb=conf/xslt/sukija.xsl=.


\subsection*{Tiedoston suggestions.xml konfigurointi}
\zlabel{suggestionConfiguration}


Tiedosto \verb|suggestions.xml| pitää konfiguroida erikseen
Sukijalle ja Voikolle. Nykyinen konfiguraatio on tehty Voikolle ja sen
vfst"-morfologialle.

Tässä vaiheessa kaikki indeksoitavista tiedostoista luetut sanat on
muutettu pieniksi kirjaimiksi eli tiedostossa
\verb|suggestions.xml| olevat erisnimetkin pitää kirjoittaa
pienellä alkukirjaimella.

Konfiguroititiedoston formaatti on määritelty tiedostossa \\
\verb=sukija-core/src/main/xsd/SuggestionInput.xsd=.

Konfigurointitiedostossa olevien säännöllisten lausekkeiden syntaksi on
sama kuin Javan luokassa \verb=java.util.regex.Pattern=

Konfiguroinnille ei ole käyttöliittymää, koska siinä voi käyttää mitä
tahansa XML"-editoria.


%%%%%%%%%%%%%%%%%%%%%%%%%%%%%%%%%%%%%%%%%%%%%%%%%%%


\bigskip
\verb|compoundWordEnd| tunnistaa yhdyssanan, jos se loppuu tiettyyn sanaan.
Esimerkiksi

\begin{verbatim}
  <compoundWordEnd>
    <input>jo(k[ie]|e) joki</input>
  </compoundWordEnd>
\end{verbatim}

tunnistaa esimerkiksi merkkijonon ''aatsajoelle''. Tällä tavalla
voidaan tunnistaa paikannimiä, jotka eivät ole sanastossa.

Jokaisessa imput-lauseessa on kaksi osaa. Ensimmäinen on säännöllinen
lauseke ja toinen jonkin sanan perusmuoto. Tunnistettaessa merkkijono
katkaistaan siitä kohdasta, josta säännöllinen lauseke alkaa, ja jos
merkkijonon loppuosan perusmuoto on argumentin toinen osa,
perusmuotona palauteaan merkkijonon alkuosa + argumentin toinen osa.

Esimeriksi ''aatsajoelle'' jaetaan kahtia osiin ''aatsa'' ja
''joelle'', ja koska merkkijonon ''joelle'' perusmuoto on ''joki'',
merkkijonon ''aatsajoelle'' perusmuodoksi tulee ''aatsajoki''.

Input-lauseita voi olla mielivaltainen määrä.


%%%%%%%%%%%%%%%%%%%%%%%%%%%%%%%%%%%%%%%%%%%%%%%%%%%


\bigskip
\verb|prefix| tunnistaa etuliitteettömän sanan (''etuliite'' voi olla
mikä tahansa merkkijono). Esimerkiksi

\begin{verbatim}
  <prefix>
    <prefix>abcdefg</prefix>
    <savePrefix>true</savePrefix>
    <saveWord>true</saveWord>
  </prefix>
\end{verbatim}

poistaa sanan alusta merkkijohon ''abcdefg'' ja yrittää tunnistaa
jäljelle jääneen sanan (''abcdefgsuomalaiselle'' => ''suomalaiselle'')
ja tallentaa sen perusmuodon (''suomalainen''). Jos \verb|savePrefix|
on true, tallentaa myös etuliitteen (''abcdefg'') ja jos
\verb|saveWord| on true, tallentaa myös koko sanan perusmuodon
('''abcdefgsuomalainen'').


%%%%%%%%%%%%%%%%%%%%%%%%%%%%%%%%%%%%%%%%%%%%%%%%%%%


\bigskip
\verb|char| muuttaa sanassa olevat merkit toiseksi. Tämä vastaa Unixin
komentoa \verb|tr|. Esimerkiksi

\begin{verbatim}
  <char>
    <from>gbdkptvw</from>
      <to>kptgbdwv</to>
  </char>
\end{verbatim}

muuttaa g:n k:ksi, b:n p:ksi jne. Ohjelma testaa muutettavien
kirjainten kaikki mahdolliset kombinaatiot. Esimerkiksi jos
tiedostosta luettu sana on ''piolokia'', komento yrittää tunnistaa
sanat ''piolokia'', ''biolokia'', ''piologia'' ja ''biologia''.


%%%%%%%%%%%%%%%%%%%%%%%%%%%%%%%%%%%%%%%%%%%%%%%%%%%


\verb|regex| muuttaa säännöllisen lausekkeen. Esimerkiksi

\begin{verbatim}
  <regex>
   <input>(ai)(j)([eou])   $1$3</input>
   <input>^([0-9]+)</input>
   <tryAll>true</tryAll>
  </regex>
\end{verbatim}

Ensimmäinen input"-lause poistaa j"-kirjaimen muun muassa sanoista
''aijemmin'', ''aijomme'' ja ''kaijutin'', ja toinen poistaa numerot
sanan alusta.


Säännöllisessä lausekkeessa voi käyttää seuraavia lyhenteitä:

\begin{verbatim}
%A  [aä]
%C  [bcdfghjklmnpqrsštvwxzž]
%O  [oö]"
%U  [uy]"
%V  [aeiouyäö]
%%  %
\end{verbatim}

Esimerkiksi \verb=<input>(%V)(h)(%V)  $1hd$3</input>=
muuttaa h:n hd:ksi esimerkiksi sanassa ''puhistus''.


Jos \verb=tryAll= on true, ohjelma kokeilee kaikkia säänöllisiä
lausekkeita, jos se on false, ohjelma lopettaa ensimmäisen tunnistetun
sanan jälkeen.

Input"-lauseen ensimmäinen osa on säännöllinen lauseke ja toinen
merkkijono, miksi se muutetaan. Sen syntaksi on sama kuin Javan
luokassa \verb=java.util.regex.Matcher=.

Jos toista osaa ei ole, säännöllisen lausekkeen tunnistama merkkijono
poistetaan. Input"-lauseita voi olla mielivaltainen määrä.


%%%%%%%%%%%%%%%%%%%%%%%%%%%%%%%%%%%%%%%%%%%%%%%%%%%


\bigskip
\verb=start= käy läpi kaikki sanan alut pisimmästä (maxLength) alkaen
lyhimpään (minLength) asti ja lopettaa, kun löytyy ensimmäinen
tunnistettu sana

\begin{verbatim}
  <start>
    <minLength>4</minLength>
    <maxLength>10</maxLength>
    <baseFormOnly>true</baseFormOnly>
  </start>
\end{verbatim}

Jos \verb=baseFormOnly= on true palautetaan sana vain, jos se on
perusmuodossa (esim ''autowerwwwww'' palauttaa ''auto''), muuten
palautetaan tunnistettu sana muutettuna perusmuotoon (''kuudenwwwww''
palauttaa ''kuusi'', mutta jos \verb=baseFormOnly= on false, ei
palauteta mitään).


\subsection*{Javalla kirjoitettu käyttöliittymä}

Kun tiedostot on indeksoitu, niitä voidaan tutkia myös komennolla

\verb|java -jar $HOME/.m2/repository/peltomaa/sukija/sukija-ui/1.1/sukija-ui-1.1.jar|

\end{document}


% cp /home/hvaisane/Lataukset/solr/solr-5.0.0/example/example-DIH/solr/db/conf/admin-extra*.html ~/Lataukset/solr/solr-5.0.0/server/solr/sukija/conf/



\bigskip
\verb|<suggestion name = "Apostrophe"/>|

Poistaa sanasta heittomerkin ja yrittää tunnistaa sanan sen jälkeen.
Jos tunnistaminen ei onnistu, poistaa sanasta heittomerkin ja kaikki
sen jälkeiset merkit ja palauttaa jäljelle jääneet merkit sanan
perusmuotona. Esimerkiksi yrittää tunnistaa merkkijonon
\verb|centime'in| muodossa \verb|centimein|. Jos tunnistaminen ei
onnistu, palauttaa merkkijonon \verb|centime|.


sudo rm -rf /var/solr/ /opt/solr* /etc/init.d/solr /etc/rc*/*solr*
