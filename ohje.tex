\documentclass[12pt]{article}
\usepackage{fontspec}
\usepackage{polyglossia}
\setdefaultlanguage{finnish}
\usepackage{csquotes}
\usepackage[lastpage,user]{zref}
\usepackage{fancyhdr}
\usepackage{setspace}
\frenchspacing
\headheight 15pt
\raggedright

\lhead{Sukijan asennus- ja käyttöohje}
\rhead{\thepage(\zpageref{LastPage})}
\cfoot{}
\begin{document}
\pagestyle{fancy}
\setlength{\parindent}{0pt}
\setlength{\parskip}{1ex plus 0.5ex minus 0.2ex}
\section*{Sukijan asennus- ja käyttöohje}


Sukija on Javalla kirjoitettu Solr:n liitännäinen suomenkielisten
tekstien indeksointiin.

Sukija analysoi sanat morfologisesti, muuttaa sanat perusmuotoon (joka
on sanakirjoissa) ja indeksoi perusmuodot, jotta sanan kaikki
taivutusmuodot löytyvät vain perusmuotoa etsimällä.

Sukija tallentaa perusmuodot Solr:n tietokantaan, josta niitä voi
etsiä Solr:n käyttöliittymän kautta.

Sukija osaa indeksoida kaikkia niitä tiedostomuotoja ja
tekstitiedostojen merkistökoodauksia, joita Apache Tika
(\verb=http://tika.apache.org/=) osaa lukea.


\subsection*{Mitä tarvitaan ja mistä ne saa?}

\begin{itemize}
\item Sukija:
      \verb=https://github.com/ahomansikka/sukija=
      Koska luet tätä tekstiä, olet jo imuroinut tämän. (-:

\item Corevoikko:
      \verb=https://github.com/voikko/corevoikko=
      Tarvitaan versiot libvoikko.so.1.14.5 ja libvoikko-4.1.1.jar

\item Solr 7.7.1:
      \verb=http://lucene.apache.org/solr/=
      Tässä dokumentissa Solr:sta käytetään nimeä solr-x.y.z,
      missä x.y.z tarkoittaa version numeroa.

\item Java versio 10. Sukijan tämä versio ei luultavasti käänny Javan
      aiemmilla versiolla.
\end{itemize}


Lisäksi Sukija tarvitsee erinäisiä jar-tiedostoja, mutta Maven imuroi
ne verkosta automaagisesti.

Tämä asennusohje olettaa, että libvoikko on asennettu hakemistoon
\verb=/usr/local/lib= ja että Solr on hakemistossa
\verb=$HOME/Lataukset/solr/solr-x.y.z=

Jos ne ovat jossain muualla, tiedostossa \verb=sukija.properties=
olevia Libvoikon tietoja pitää muuttaa vastaavasti.


\subsection*{Solr:n asentaminen}

Solr:ia ei tarvitse asentaa, vaan sitä voi käyttää suoraan
hakemistosta \verb=$HOME/Lataukset/solr/solr-x.y.z=.

Tuotantokäyttöä varten Solr on parasta asentaa palveluna (service).
Tämä ohje olettaa, että Solr on asennettu palveluna.

Solr asennetaan palveluna skriptillä
\verb=install_solr_service.sh= 
(katso Apache Solr Reference Guide
kohta Service Installation Script).

Jos et malta katsoa, se tehdään hakemistossa
\verb=$HOME/Lataukset/solr/= tällä tavalla

{\footnotesize
\verb=sudo bash solr-x.y.z/bin/install_solr_service.sh solr-x.y.z.tgz -u username=
}

missä \verb=username= on oma käyttäjätunnuksesi.
Jos kohdan \verb=-u username= jättää pois, Solr asennetaan tunnuksen
\verb=solr= alle.

Skripti asentaa Solr:n hakemistoihin \verb=/opt/solr= ja
\verb=/var/solr=.


\subsection*{Sukijan kääntäminen ja asentaminen}

Sukija käännetään ja asennetaan komennolla

\verb=mvn install=

Komento imuroi netistä tarvitsemansa Javan jar-paketit eli ensimmäinen
kääntäminen saattaa kestää kauan. Erityisen kauan se kestää, jos et
ole aiemmin käyttänyt mavenia.

Sukijan jar-tiedosto asennetaan maven-hakemistoon \verb|${HOME}/.m2|


\subsection*{Sukijan konfigurointi}

Sukijan parametrit asetetaan tiedostossa \verb|sukija.properties|,
jossa on myös tarvittavat ohjeet.

Sen jälkeen komennolla

\verb|make tiedostot|

tehdään Solr:n asetustiedostot \verb=conf/data-config.xml= ja
\verb=conf/schema.xml=. Esimerkit näistä tiedostoista
saa aikaan yllä mainitulla komennolla. (-:

Seuraavaksi pitää tehdä Solr:iin Sukijan tarvitsemat hakemistot ja
kopioida niihin Solr:n tarvitsemat tiedostot. Sitä varten on kaksi
skriptiä

\verb=src/main/resources/install.sh=
\verb=src/main/resources/install-sudo.sh=

Jälkimmäistä käytetään, jos Solr on asennettu tunnuksella \verb=solr=.

Jos asentamisen jälkeen konfiguroi Sukijaa, muutokset voi viedä
Solr:iin jommallakummalla komennoista

\verb=cp -r conf/* /var/solr/data/sukija/conf/=
\verb=sudo -u solr cp -r conf/* /var/solr/data/sukija/conf/=


Seuraavaksi mennään selaimella osoitteeseen \verb=http://localhost:8983/solr/=
Vasemmalla paneelin alalaidassa lukee 

\begin{verbatim}
No cores available
Go and create one
\end{verbatim}

Mennään sinne ja kirjoitetaan kohtaan \verb=name= ja \verb=instanceDir=
''sukija'' (mutta ilman lainausmerkkejä) ja napsautetaan kohtaa
\verb=Add Core=.

Nyt vasemmalle paneeliin alalaitaan pitäisi tulla mahdollisuus valita
indeksi (Solr käyttää siitä nimeä \verb=core=) sukija. Sen jälkeen voi
ruveta indeksoimaan. Katso sivu \zpageref{indexing}.


\subsection*{Sukijan loki}

Sukijan lokitulostus konfiguroidaan Solr:n lokitulostuksen kautta
tiedostossa \verb=/var/solr/log4j2.xml=. Sukijan käyttämät lisäykset
ovat on hakemistossa \verb=src/main/resources/log4j2.xml=. Ne on
kommentoitu pois, mutta niitä voi siirtää halutun määrän kommenttien
ulkopuolelle. Jotta Solr näkisi muutokset, ne pitää laittaa tiedostoon
\verb=/var/solr/log4j2.xml=.

Lokitulostuksen eri tasot (\verb=ALL=, jne.) voi katsoa Solr:n
dokumenteista.

Lokitulostus menee kansioon \verb=/var/solr/logs/=


\subsection*{Indeksointi}
\zlabel{indexing}

Tiedostot indeksoidaan menemällä osoitteeseen
\verb|http://localhost:8983/solr/sukija/dataimport?command=full-import|

Indeksoinnin voi konfiguroida ja aloittaa myös Solr:n käyttöliittymästä
kohdassa \verb|Dataimport|.

Enemmän tai vähemmän pitkän ajan kuluttua indeksoinnin lopputuloksen
voi katsoa osoitteesta
\verb|http://localhost:8983/solr/sukija/dataimport|
tai Solr:n käyttöliittymästä.


\subsection*{Tietojen etsiminen}

Sanoja etsitään menemällä osoitteeseen
\verb=http://localhost:8983/solr/sukija/browse=

Etsittävien sanojen tulee olla perusmuodossa. Etsittäessä sanoja ei
muuteta perusmuotoon siksi, että yhden sanan perusmuoto voi olla
toisen sanan taivutusmuoto. Paras esimerkki tästä on ''alusta'', joka
on sanojen ''alusta'', ''alustaa'', ''alku'', ''alunen'' ja ''alus''
taivutusmuoto. Tällöin herää kysymys, mitä sanaa pitää etsiä, vai
etsitäänkö kaikkia?


Eri tavalla muotoillun tulostuksen saa osoitteesta
\verb=http://localhost:8983/solr/sukija/select=

Esimerkiksi sanaa \verb=sana= etsitään näin:
\verb|http://localhost:8983/solr/sukija/select?q=sana|

Tämän tulostuksen ulkonäköä voi muuttaa muuttamalla Sukijan mukana
tulevaa tiedostoa \verb=conf/xslt/sukija.xsl=.

Tietoja voi etsiä myös Solr:n käyttöliittymällä kohdassa \verb|Query|.


\subsection*{Tiedoston suggestions.xml konfigurointi}
\zlabel{suggestionConfiguration}

Tässä vaiheessa kaikki indeksoitavista tiedostoista luetut sanat on
muutettu pieniksi kirjaimiksi eli tiedostossa
\verb|suggestions.xml| olevat erisnimetkin pitää kirjoittaa
pienellä alkukirjaimella.

Konfigurointitiedoston formaatti on määritelty tiedostossa
\verb=/src/main/xsd/SuggestionInput.xsd=.

Konfigurointitiedostossa olevien säännöllisten lausekkeiden syntaksi on
sama kuin Javan luokassa \verb=java.util.regex.Pattern=

Konfiguroinnille ei ole käyttöliittymää, koska tiedoston
\verb=suggestions.xml= muokkaamiseen voi käyttää mitä tahansa
XML-editoria.


%%%%%%%%%%%%%%%%%%%%%%%%%%%%%%%%%%%%%%%%%%%%%%%%%%%


\bigskip
\verb|compoundWordEnd| tunnistaa yhdyssanan, jos se loppuu tiettyyn sanaan.
Esimerkiksi

\begin{verbatim}
  <compoundWordEnd>
    <input>joki joki</input>
    <input>joke joki</input>
    <input>joe joki</input>
  </compoundWordEnd>
\end{verbatim}

tunnistaa esimerkiksi merkkijonon ''aatsajoelle''. Tällä tavalla
voidaan tunnistaa paikannimiä, jotka eivät ole sanastossa.

Jokaisessa input-lauseessa on kaksi osaa. Ensimmäinen on jokin
merkkijono ja toinen jonkin sanan perusmuoto. Tunnistettaessa
merkkijono katkaistaan siitä kohdasta, josta ensimmäinen merkkijono
alkaa, ja jos merkkijonon loppuosan perusmuoto on argumentin toinen
osa, perusmuotona palautetaan merkkijonon alkuosa + argumentin toinen
osa.

Esimeriksi ''aatsajoelle'' jaetaan kahtia osiin ''aatsa'' ja
''joelle'', ja koska merkkijonon ''joelle'' perusmuoto on ''joki'',
merkkijonon ''aatsajoelle'' perusmuodoksi tulee ''aatsajoki''.

Input-lauseita voi olla mielivaltainen määrä.

Ensimmäinen merkkijono ei ole säännöllinen lauseke, koska ne ovat
hitaampia kuin merkkijonot varsinkin, kun merkkijonojen tunnistuksessa
käytetään Aho-Corasick -algoritmia.\footnote{Katso
Alfred V. Aho and Margaret J. Corasick:
\emph{Efficient String Matching: An Aid to Bibliographic Search.}
Communications of the ACM. June 1975 Volume 18 Number 6.
Löytyy netistä googlaamalla.} Algoritmissa sanan (ensimmäisen
merkkijonon) etsimiseen käytetty aika ei riipu etsittävien sanojen
määrästä, sillä se etsii kaikkia sanoja samanaikaisesti.

%%%%%%%%%%%%%%%%%%%%%%%%%%%%%%%%%%%%%%%%%%%%%%%%%%%


\bigskip
\verb|prefix| tunnistaa etuliitteettömän sanan (''etuliite'' voi olla
mikä tahansa merkkijono). Esimerkiksi

\begin{verbatim}
  <prefix>
    <prefix>abcdefg</prefix>
    <savePrefix>true</savePrefix>
    <saveWord>true</saveWord>
  </prefix>
\end{verbatim}

poistaa sanan alusta merkkijonon ''abcdefg'' ja yrittää tunnistaa
jäljelle jääneen sanan (''abcdefgsuomalaiselle'' => ''suomalaiselle'')
ja tallentaa sen perusmuodon (''suomalainen''). Jos \verb|savePrefix|
on true, tallentaa myös etuliitteen (''abcdefg'') ja jos
\verb|saveWord| on true, tallentaa myös koko sanan perusmuodon
('''abcdefgsuomalainen'').

Prefix-lauseita voi olla mielivaltainen määrä.

%%%%%%%%%%%%%%%%%%%%%%%%%%%%%%%%%%%%%%%%%%%%%%%%%%%


\bigskip
\verb|char| muuttaa sanassa olevat merkit toiseksi. Tämä vastaa Unixin
komentoa \verb|tr|. Esimerkiksi

\begin{verbatim}
  <char>
    <from>gbdkptvw</from>
      <to>kptgbdwv</to>
  </char>
\end{verbatim}

muuttaa g:n k:ksi, b:n p:ksi jne. Ohjelma testaa muutettavien
kirjainten kaikki mahdolliset kombinaatiot. Esimerkiksi jos
tiedostosta luettu sana on ''piolokia'', komento yrittää tunnistaa
sanat ''piolokia'', ''biolokia'', ''piologia'' ja ''biologia'' (mutta
ei välttämättä tässä järjestyksessä :-).


%%%%%%%%%%%%%%%%%%%%%%%%%%%%%%%%%%%%%%%%%%%%%%%%%%%


\verb|regex| muuttaa säännöllisen lausekkeen. Esimerkiksi

\begin{verbatim}
  <regex>
   <input>(ai)(j)([eou])   $1$3</input>
   <input>^([0-9]+)</input>
   <tryAll>true</tryAll>
  </regex>
\end{verbatim}

Ensimmäinen input-lause poistaa j-kirjaimen muun muassa sanoista
''aijemmin'', ''aijomme'' ja ''kaijutin'', ja toinen poistaa numerot
sanan alusta.


Säännöllisessä lausekkeessa voi käyttää seuraavia lyhenteitä:

\begin{verbatim}
%A  [aä]
%C  [bcdfghjklmnpqrsštvwxzž]
%O  [oö]
%U  [uy]
%V  [aeiouyäö]
%%  %
\end{verbatim}

Esimerkiksi \verb=<input>(%V)(h)(%V)  $1hd$3</input>=
muuttaa h:n hd:ksi esimerkiksi sanassa ''puhistus''.


Jos \verb=tryAll= on true, ohjelma kokeilee kaikkia säänöllisiä
lausekkeita, jos se on false, ohjelma lopettaa ensimmäisen tunnistetun
sanan jälkeen.

Input-lauseen ensimmäinen osa on säännöllinen lauseke ja toinen
merkkijono, miksi se muutetaan. Sen syntaksi on sama kuin Javan
luokassa \verb=java.util.regex.Matcher=.

Jos toista osaa ei ole, säännöllisen lausekkeen tunnistama merkkijono
poistetaan. Input-lauseita voi olla mielivaltainen määrä.


%%%%%%%%%%%%%%%%%%%%%%%%%%%%%%%%%%%%%%%%%%%%%%%%%%%


\bigskip
\verb=start= käy läpi kaikki sanan alut pisimmästä (maxLength) alkaen
lyhimpään (minLength) asti ja lopettaa, kun löytyy ensimmäinen
tunnistettu sana

\begin{verbatim}
  <start>
    <minLength>4</minLength>
    <maxLength>10</maxLength>
    <baseFormOnly>true</baseFormOnly>
  </start>
\end{verbatim}

Jos \verb=baseFormOnly= on true palautetaan sana vain, jos se on
perusmuodossa (esim. ''autowerwwwww'' palauttaa ''auto''), muuten
palautetaan tunnistettu sana muutettuna perusmuotoon (''kuudenwwwww''
palauttaa ''kuusi'', mutta jos \verb=baseFormOnly= on false, ei
palauteta mitään).


%%%%%%%%%%%%%%%%%%%%%%%%%%%%%%%%%%%%%%%%%%%%%%%%%%%


\bigskip

\verb=apostrophe= poistaa sanasta heittomerkin ja yrittää tunnistaa
sanan sen jälkeen. Jos tunnistaminen ei onnistu, poistaa sanasta
heittomerkin ja kaikki sen jälkeiset merkit ja palauttaa jäljelle
jääneet merkit sanan perusmuotona. Esimerkiksi yrittää tunnistaa
merkkijonon \verb=centime'in= muodossa \verb=centimein=. Jos
tunnistaminen ei onnistu, palauttaa merkkijonon \verb=centime=.


\begin{verbatim}
  <apostrophe/>
\end{verbatim}

Tällä komennolla ei ole parametreja.


\bigskip \noindent \noindent
Copyright © 2011--2018 Hannu Väisänen.
\end{document}



\subsection*{Javalla kirjoitettu käyttöliittymä}

Kun tiedostot on indeksoitu, niitä voidaan tutkia myös komennolla

{\footnotesize
\verb|java -jar $HOME/.m2/repository/peltomaa/sukija/sukija-ui/1.1/sukija-ui-1.1.jar|
}

\bigskip \noindent \noindent
Copyright © 2011--2018 Hannu Väisänen.
\end{document}


% cp /home/hvaisane/Lataukset/solr/solr-x.y.z/example/example-DIH/solr/db/conf/admin-extra*.html ~/Lataukset/solr/solr-5.0.0/server/solr/sukija/conf/


sudo rm -rf /var/solr/ /opt/solr* /etc/default/solr.in.sh /etc/init.d/solr /etc/rc*/*solr*
