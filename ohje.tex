\documentclass[12pt,a4paper]{scrartcl}
\usepackage[finnish]{babel}
\usepackage[T1]{fontenc}
\usepackage[utf8]{inputenc}
\usepackage{zref-lastpage,zref-user}
\usepackage{fancyhdr}
\frenchspacing
\lhead{Sukijan asennus- ja käyttöohje (Copyright © 2011--2013 Hannu Väisänen)}
\rhead{\thepage(\zpageref{LastPage})}
\cfoot{}
\begin{document}
\pagestyle{fancy}
\setlength{\parindent}{0pt}
\setlength{\parskip}{1ex plus 0.5ex minus 0.2ex}
\section*{Sukijan asennus- ja käyttöohje}

Sukija on Javalla kirjoitettu ohjelma suomenkielisten tekstien
indeksointiin.

Sukija analysoi sanat morfologisesti, muuttaa sanat perusmuotoon (joka
on sanakirjoissa) ja indeksoi perusmuodot, jotta sanan kaikki
taivutusmuodot löytyvät vain perusmuotoa etsimällä.

Sukija tallentaa perusmuodot Solr:n tietokantaan, josta niitä voi
etsiä Solr:n käyttöliittymän kautta.

Sukija osaa indeksoida kaikkia niitä tiedostomuotoja, joita Apache
Tika \\(http://tika.apache.org/) osaa lukea.

\subsection*{Mitä tarvitaan ja mistä ne saa?}

\begin{itemize}

\item Sukija \\
      https://github.com/ahomansikka/sukija \\
      Koska luet tätä tekstiä, olet jo imuroinut tämän. (-:

\item Suomi-Malaga \\
      Se on corevoikossa (https://github.com/voikko/corevoikko)
      hakemistossa \\
      suomimalaga.

\item Apache Solr 4.1.0 \\
      http://lucene.apache.org/solr/

\item Ubuntun paketit libmalaga7 ja maven.
\end{itemize}

Lisäksi Sukija tarvitsee erinäisiä jar-tiedostoja, mutta Maven imuroi
ne verkosta automaagisesti.

Sukijaa voi käyttää myös Voikon Java-version kanssa. Tällöin tarvitaan
myös Ubuntun paketti libvoikko1.

Tämä asennusohje olettaaa, että corevoikko ja apache-solr ovat hakemistoissa \\
\verb=$HOME/Lataukset/corevoikko/= ja \\
\verb=$HOME/Lataukset/solr-4.1.0=

\newpage
\subsection*{Ohjelman rakenne}

Sukijassa on kolme osaa:

\begin{itemize}
\item sukija-core     Java-luokkia, joita muut ohjelman osat tarvitsevat.
\item sukija-malaga   Solr:n liitännäinen, joka käyttää Suomi-Malagan
                      Sukija-versiota \\muuttamaan sanat perusmuotoon.
\item sukija-voikko   Solr:n liitännäinen, joka käyttää Voikkoa
                      muuttamaan sanat perusmuotoon.
\end{itemize}


\subsection*{Suomi-Malagan asentaminen}

Suomi-Malagasta on kaksi versiota, Voikko-versio on tarkoitettu
oikolukuun ja Sukija tiedostojen indeksointiin. Sukija-versio
käännetään komennolla

\begin{verbatim}
cd $HOME/Lataukset/corevoikko/suomimalaga
make sukija
\end{verbatim}

Tee alihakemisto \verb=$HOME/.sukija= ja kopioi sinne tiedostot \\
\verb=suomimalaga/sukija/{suomi.*_l,suomi.pro}=

Myös Voikko-versiota voi käyttää indeksointiin, kun sen kääntää ja
asentaa komennoilla

\begin{verbatim}
cd $HOME/Lataukset/corevoikko/suomimalaga
make voikko-sukija
make voikko-install DESTDIR=~/.voikko
\end{verbatim}

\verb|DESTDIR| voi olla myös joitan muuta kuin \verb|~/.voikko|.

Versioiden erot ovat siinä, että Sukija"-versio tunnistaa myös vanhoja
taivutusmuotoja ja sanoja sekä yleisiä kirjoitusvirheitä.


\subsection*{Sukijan kääntäminen ja asentaminen, Solr:n konfigurointi}

Ensin käännetään Sukija komennolla

\verb=mvn package=

Komento imuroi netistä tarvitsemansa Javan jar-paketit eli ensimmäinen
kääntäminen saattaa kestää kauan. Erityisen kauan se kestää, jos et
ole aiemmin käyttänyt mavenia.

\bigskip Toisessa vaiheessa asetetaan Solr:n konfigurointitiedostoon
\verb=schema.xml= saneistajaluokka (ulkomaankielellä tokenizer), joka
lukee sanat tiedostoista, ja morfologialuokka, joka muuttaa sanat
perusmuotoon. Komennolla \verb|make ____-schema| on viisi eri
vaihtoehtoa:

\begin{tabular}{@{}ll}
Komento                              & Morfologialuokka \\
\verb=make malaga-schema=            & MalagaMorphologyFilterFactory \\
\verb=make malaga-suggestion-schema= & MalagaMorphologySuggestionFilterFactory \\
\verb=make voikko-schema=            & VoikkoMorphologyFilterFactory \\
\verb=make voikko-suggestion-schema= & VoikkoMorphologySuggestionFilterFactory \\
\verb=make debug-schema=             &
\end{tabular}

Komentoa \verb=make debug-schema= käytetään vain Sukijan
kehittämiseen.

Saneistajan oletusarvo on FinnishTokenizerFactory, joka tulee Sukijan
mukana, mutta sen voi vaihtaa muuttujalla TOKENIZER\_FACTORY
esimerkiksi näin:

\verb|make voikko-schema TOKENIZER_FACTORY=solr.StandardTokenizerFactory|

\verb=____FilterFactory= ja \verb=____SuggestionFilterFactory= eroavat
toisistaan siten, että jos morfologialuokka ei tunnista sanaa,
\verb=Suggestion="-luokissa sanaan tehdään muutoksia (esimerkiksi
muutetaan w v:ksi) ja tunnistusta yritetään uudestaan. Tämä ei ole
sama asia kuin Voikon oikeinkirjoituksen korjausehdotukset!

\verb=Suggestion="-luokat konfiguroidaan tiedostossa
\verb|suggestion.txt|. Katso sivu \zpageref{suggestionConfiguration}.

\bigskip
Indeksoitavat tiedostot asetetaan tiedostossa \verb|data-config.xml|.

Katso \\
\verb|http://wiki.apache.org/solr/DataImportHandler#FileListEntityProcessor| \\
ja \\
\verb|http://wiki.apache.org/solr/TikaEntityProcessor|).

Tärkeimmät konfiguroitavat parametrit ovat:

\begin{itemize}
\item baseDir
Hakemisto, jossa ja jonka alihakemistoissa tiedostot ovat.

\item fileName
Säännöllinen lauseke, joka valitsee indeksoitavat tiedostot.

\item excludes
Säännöllinen lauseke, joka valitsee tiedostot, joita ei indeksoida.
\end{itemize}

Komento \verb|make install| asettaa näiden oletusarvoiksi

baseDir \verb|$HOME/Asiakirjat|

fileName \verb|.*| eli kaikki tiedostot indeksoidaan.

excludes \\
{\footnotesize \verb+(?u)(?i).*[.](au|bmp|bz2|class|gif|gpg|gz|jar|jpg|jpeg|m|o|png|tif|tiff|wav|zip)$+}


Näitä voidaan muuttaa parametreilla \verb|BASE_DIR|,
\verb|FILE_NAME| ja
\verb|EXCLUDES|. Esimerkiksi

\verb|make install BASE_DIR=/usr/local/data|

Indeksoitavien tiedostojen asettamisen lisäksi komento
\verb=make install= kopioi hakemistossa \verb=conf= olevat Solr:n ja
Sukijan tarvitsemat tiedostot oikeisiin paikkoihin.

Tiedostot
\verb=data-config.xml=,
\verb=logging.properties=,
\verb=suggestion.txt= ja
\verb=synonyms.txt=
kopioidaan hakemistoon \verb=$HOME/.sukija=
ja tiedostot
\verb=schema.xml=,
\verb=solrconfig.xml=, \\
\verb=sukija-context.xml= ja
\verb=sukija.xsl=
niihin hakemistoihin, joista Solr löytää ne.


\subsection*{Solr:n käynnistäminen}

\begin{verbatim}
cd $HOME/Lataukset/solr-4.1.0/example
java -jar start.jar
\end{verbatim}

Jos Solr valittaa jna:sta, käynnistyskomento on

\verb|java -Djna.nosys=true -jar start.jar|

Solr:n lokia voi konfiguroida lisäämällä Solr:n käynnistykseen:

\verb|-Djava.util.logging.config.file=$HOME/.sukija/logging.properties|


Javan lokin (java.util.logging) konfigurointi on oma taiteenlajinsa,
ja jotta asia ei olisi liian yksinkertainen, Solr ja niin ollen myös
Sukijan Solr-liitännäiset käytävät SLF4J:tä (www.slf4j.org), joka
puolestaan käyttää Javan lokisysteemiä.

Solr:n käynnistymisen voi varmistaa selaimessa menemällä verkko-osoitteeseen

\verb|http://localhost:8983/solr/admin/|


\subsection*{Indeksointi}

Tiedostot indeksoidaan menemällä osoitteeseen

\verb|http://localhost:8983/solr/dataimport?command=full-import|

Enemmän tai vähemmän pitkän ajan kuluttua indeksoinnin lopputuloksen
voi katsoa osoitteesta

\verb|http://localhost:8983/solr/dataimport|



\subsection*{Tiedoston suggestion.txt konfigurointi}
\zlabel{suggestionConfiguration}

Tiedosto \verb|suggestion.txt| pitää konfiguroida erikseen Sukijalle
ja Voikolle. Nykyinen konfiguraatio on tehty Sukijalle.

Konfiguraatiossa on neljä komentoa.

\bigskip
\verb|Apostrophe| Poistaa sanasta heittomerkin ja yrittää tunnistaa
sanan sen jälkeen. Jos tunnistaminen ei onnistu, poistaa sanasta
heittomerkin ja kaikki sen jälkeiset merkit ja palauttaa jäljelle
jääneet merkit sanan perusmuotona. Esimerkiksi \verb|Bordeaux'iin|
yritetään tunnistaa muodossa \verb|Bordeauxiin|. Jos sitä ei
tunnisteta, palauttaa merkkijonon \verb|Bordeaux|.

\bigskip
\verb|Char| Muuttaa sanassa olevan yhden merkin toiseksi. Esimerkiksi

\verb|Char "w" "v"|

muuttaa w:n v:ksi (''wanha'' => ''vanha'').

\bigskip
\verb|CharCombination| Muuttaa yhden tai usemman merkin kaikki
kombinaatiot. Esimerkiksi

\verb|CharCombination "pk" "bg"|

(1) muuttaa p:t b:iksi, jättää k:t ennalleen,
(2) muuttaa k:t g:iksi, jättää p:t ennalleen, sekä
(3) muuttaa p:t b:iksi ja k:t g:iksi.

Siis jos tiedostosta luettu sana on ''piolokia'', komento muuttaa sen
sanoiksi ''biolokia'', ''piologia'' ja ''biologia''.


\bigskip
\verb|Length3| poistaa kolmesta peräkkäisestä samasta kirjaimesta
yhden. Esimerkiksi ''kauttta'' => ''kautta''.


\bigskip
\verb|Regex| muuttaa säännöllisen lausekkeen. Esimerkiksi

\verb|Regex "ai(j)[eou]" ""|

poistaa j"-kirjaimen muun muassa sanoista ''aijemmin'', ''aijomme'' ja
''kaijutin''.


Säännöllisessä lausekkeessa voi käyttää kirjainta \verb|C|
tarkoittamaan konsonantteja ja kirjainta \verb|V| tarkoittamaan
vokaaleja. Esimerkiksi

\verb|Regex  "C[ae](hi)C"  "i"|

poistaa h"-kirjaimet muun muassa sanoista ''ainahinen'' ja
''etehinen''.

\bigskip
Näitä komentoja voi antaa mielivaltaisen paljon missä tahansa
järjestyksessä, ja \\ \verb|____SuggestionFilterFactory| palauttaa
perusmuotona ensimmäisen tunnistamansa muodon. Jos mitään ehdotusta ei
tunnisteta, \verb|____SuggestionFilterFactory| palauttaa alkuperäisen
merkkijonon.

Tiedostossa \verb|suggestion.txt| voi olla tyhjiä rivejä. Kommentti
alkaa merkillä \verb|#| ja jatkuu rivin loppuun.

\end{document}
